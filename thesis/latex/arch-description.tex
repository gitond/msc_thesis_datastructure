\chapter{Architecture Description} \label{Arch description}

\section{Problem Description} \label{probDesc}

The main goal of this thesis is to develop an application which aims to use a 
modern CV and DL-based tracking solution in an AR application. This is first 
defined in \hyperref[rq2]{\textbf{RQ2}}. Here we further define the needs of 
this prototype. The aim of this prototype is to guide the user through the 
preparation of a single recipe. The goal would be to offer real-time guidance 
for all the actions involved in making the chosen dish. AR-augmentations 
would need to be created for each of the actions involved. Examples of these 
actions are for example adding ingredients to pots, handling kitchen 
appliances, peeling and chopping ingredients and waiting for cooking times. \par
	There is a case to be made here for using more advanced CV and 
DL-based methods for tracking. Most obvious among these is the simple fact 
that there are lots of things to track. Just in the case of a simple pasta 
Bolognese recipe one needs to track a bunch of different ingredients, such as 
onion, meat, tomato sauce etc. These also have different states, such as raw, 
peeled, chopped, translucent in the case of the onion. On top of ingredients, 
different kitchenware such as pots and pans, and also different kitchen 
appliances need to be tracked. An example would be an electric stove top with 
different heat settings. \par
	Cooking was chosen as the example use-case because it is a real world 
problem where offering AR-based guidance would be useful and where using 
advanced CV-methods in tracking is justified. In \ref{protos} we also see 
prototypes developed by other teams using a similar step-by-step tutorial 
format for their AR-content. Both \textcite{pylvanainen} and 
\textcite{reyesEtAl2016} use this kind of approach to teach instrument use in 
laboratories. But their applications use traditional tracking methods that 
are developed for their specific instruments. If an easily retrainable modern 
CV-based tracking method is used, one could theoretically take the cooking 
app developed for this prototype, retrain the tracking module and use it with 
any step-by-step tutorial, making the potential of this sort of application 
huge.

\section{Perceived Challenges} \label{challenges}

\begin{itemize}
	\item{Do this based on \ref{Literature review}}
	\item{Mention challenges encountered by others & possible solutions if needed}
	\item{Add as many subsections as needed}
\end{itemize}

\section{Proposed Architecture} \label{arch}

\begin{figure}
\centering \includegraphics[width=0.5\textwidth]{kuvat/placeholder.png}
\caption{Visual Representation of the Proposed Architecture}
\label{Arch fig} 
\end{figure}

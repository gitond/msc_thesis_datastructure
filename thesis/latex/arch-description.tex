\chapter{Architecture Description} \label{Arch description}

\section{Problem Description} \label{probDesc}

% Justication for RQ2 should be included in this chapter
%%%
% RQ2: Can a system with a backend computer vision system and an AR user 
% interface be used in a cooking environment?
%%%
%  - Goal: build a demo system thet uses CV with AR
In order to empirically examine the use of an AR application with a modern CV 
and DL-based tracking solution in a cooking environment, as defined in 
\hyperref[rq2]{\textbf{RQ2}}, a suitable prototype needs to be developed. In 
this chapter we aim to define this prototype: exactly what it'll need to do and 
what kind of challenges we might face during development. In chapter 
\ref{implementation} we further describe the steps taken during development 
and the finished prototype. \par
% Issue #29 Prototype Description
% - Our prototype will guide the user through the recipe...
% - Computer Vision is used to...
% - To test the prototype we'll
	The application developed for this thesis aims to guide the user 
through the recipe defined in listing \ref{recipelisting}. In it Computer 
Vision is used to track bowls, spoons, knives, apples, oranges and bananas. 
In chapter \ref{usability}, the application will be used to conduct a usability 
test, where subjects are asked to prepare the recipe with the help of the 
application and their user experience is measured in a survey. \par
% Justifying CV & discussing our recipe
%%%
% - Goal: offer \textbf{real-time} guidance for \textbf{actions} involved in
%   making the dish
% - adding ingredients to pots, handling kitchen appliances, peeling and 
%   chopping ingredients
% - These actions as AR-annotations (animations)
% - Case for using CV in tracking: lots of things to track
% - Ingredients
%   - These have different states
% - Different kitchenware
%   - bowl, knives
%%%
	The goal would be to offer real-time guidance for all of the actions 
involved in making the fruit salad as defined in listing \ref{recipelisting}. 
AR-augmentations would need to be created for each of the actions involved. 
Examples of these actions are chopping the fruits and placing them in the 
bowl.\par
	There is a case to be made here for using more advanced CV and 
DL-based methods for tracking. Most obvious among these is the simple fact 
that there are lots of things to track. Just in the case of our simple fruit 
salad, one needs to track three different ingredients, which physically change 
shape as they are chopped into pieces during the preparation process. On top 
of the ingredients, different kitchenware such as a bowl and at least one knife
needs to be tracked. It is pretty clear that such an application with this 
many different trackable objects and changing states would be extremely 
difficult to implement using traditional AR tracking methodologies, thus CV 
and DL need to be used. \par
% Why cooking?
%%%
% - Real world problem where CV with AR could be useful.
% - Similar step-by-step approach as other prototypes in the education field
% - pylvanainen, reyesEtAl2016: step-by-step instrument teaching tutorial 
%   approach
% - If the end product here is successful the general approach could be 
%   adapted to different fields
%%%
	Cooking was chosen as the example use-case because it is a real world 
problem where offering AR-based guidance would be useful and where using 
advanced CV-methods in tracking is justified. In \ref{protos} we also see 
prototypes developed by other teams using a similar step-by-step tutorial 
format for their AR-content. Both \textcite{pylvanainen} and 
\textcite{reyesEtAl2016} use this kind of approach to teach instrument use in 
laboratories, but their applications use traditional tracking methods that 
are developed for their specific instruments. If an easily retrainable modern 
CV-based tracking method is used, like in our application, one could 
theoretically just retrain the tracking module, and use it in a completely 
different environment, with any step-by-step tutorial, making the potential of 
this sort of application huge. \par
% - We want our prototype to run on mobile
% - We want our prototype to be easily portable between platforms
	In order for our prototype to be usable in a kitchen environment it 
needs to be able to run on a mobile phone. However the prototype has to be 
easily portable to other platforms in the future. 
% Issue #32 how to run on mobile
% - At this time we are focused on building a web app
For this reason it will be developed as a web app. Testing the application 
will be performed on a mobile device.

\section{Perceived Challenges} \label{challenges}

The first challenge in building the prototype is defining a recipe and 
deciding what to track. If there are too many things to track the complexity 
increases beyond the scope of a thesis. However if there are not enough 
things to track then the prototype will fail to demonstrate the potential of 
CV and AR in solving real world problems and thus be useless.

\section{Proposed Architecture} \label{arch}

\begin{figure}
\centering \includegraphics[width=0.5\textwidth]{kuvat/placeholder.png}
\caption{Visual Representation of the Proposed Architecture}
\label{Arch fig} 
\end{figure}

\chapter{Introduction} \label{intro}

\section{The goals of this thesis} \label{goals}
This thesis aims to design and implement a system that integrates a backend 
powered computer vision with an Augmented Reality (AR) interface. The concept 
involves a device capturing an image feed through its camera, which is then 
transmitted to a processing unit. Here, computer vision algorithms analyze 
the data to extract meaningful information, which is subsequently sent to the 
AR interface to give feedback to user. In this thesis, I document the journey 
of developing this system and assess its performance.

\section{Research Questions} \label{rq}
The thesis aims to answer the following Research Questions:
\begin{description}
	\item[RQ1:] \label{rq1} What are the technological challenges in 
	combining advanced computer vision algorithms with an AR user interface?

	\item[RQ2:] \label{rq2} Can a system with a backend computer vision 
	system and an AR user interface be used in a cooking environment?

	\item[RQ3:] \label{rq3} Can such a system provide satisfactory user 
	experience?
\end{description}

\section{Methodology Overview} \label{meth}
The work covered in this thesis mostly consist of the following three phases: 
firstly I conduct a literature review in chapter \ref{Literature review}. The 
purpose of this is to learn about the technologies involved in this project, 
to find out what perceived challenges were found by other people working with 
these technologies and to search for projects similar to ours, conducted by the 
scientific community. Starting with a literature review should also provide a 
firm scientific basis to the later phases. \par
	The second phase of the work concerns architecture design and prototype 
development. All relevant technological challenges found during the literature 
review are collected to chapter \ref{challenges} for further analysis. Based 
on all these findings we then propose an architecture for a prototype in 
chapter \ref{arch}. The prototype is then implemented and the whole 
development process, the technologies used, as well as anything notable that 
happens during development is described in chapter \ref{implementation}. \par
	The third phase of the work is an empirical usability study conducted 
on the developed prototype. Here the finished prototype is given out to test 
subjects to measure how well the system performs. Usability is measured both 
through asking the opinions of the test subjects through a questionnaire and 
through measurements made by the prototype software. All the collected data and 
the questionnaire used can be found in the attachments. This phase of work is 
more thoroughly described in the chapters \ref{usability} and \ref{feasibility}.

\chapter{Introduction} \label{intro}

\section{The goals of this thesis} \label{goals}
This thesis explores the usability and functionalities of an Augmented Reality 
(AR) experience and its interoperability with modern Deep Learning (DL) based 
computer vision (CV). This is explored with a prototype, where a camera is 
used to capture an image feed, information is extracted from it using 
CV-algorithms and finally AR is used to communicate this information with the 
user. This thesis explores the creation of such a system and examines its 
usability and potential real-world usefulness.

\section{Research Questions} \label{rq}
The thesis aims to answer the following Research Questions:
\begin{description}
	\item[RQ1:] \label{rq1} What are the technological challenges in 
	combining advanced computer vision algorithms with an AR user interface?

	\item[RQ2:] \label{rq2} Can a system with a backend computer vision 
	system and an AR user interface be used in a cooking environment?

	\item[RQ3:] \label{rq3} Can such a system provide satisfactory user 
	experience?
\end{description}

\section{Methodology Overview} \label{meth}
The work covered in this thesis mostly consist of the following three phases: 
firstly I conduct a literature review in chapter \ref{Literature review}. The 
purpose of this is to learn about the technologies involved in this project, 
to find out what perceived challenges were found by other people working with 
these technologies and to search for projects similar to ours, conducted by the 
scientific community. Starting with a literature review should also provide a 
firm scientific basis to the later phases. In this phase we aim to answer 
\hyperref[rq1]{\textbf{RQ1}}.\par
	The second phase of the work concerns architecture design and prototype 
development. All relevant technological challenges found during the literature 
review are collected to chapter \ref{challenges} for further analysis. Based 
on all these findings we then propose an architecture for a prototype in 
chapter \ref{arch}. The prototype is then implemented and the whole 
development process, the technologies used, as well as anything notable that 
happens during development is described in chapter \ref{implementation}. The 
goal of this phase is to build an actual prototype and thus answer 
\hyperref[rq2]{\textbf{RQ2}}.\par
	The third phase of the work is an empirical usability study conducted 
on the developed prototype. Here the finished prototype is given out to test 
subjects to measure how well the system performs. Usability is measured both 
through asking the opinions of the test subjects through a questionnaire and 
through measurements made by the prototype software. All the collected data and 
the questionnaire used can be found in the attachments. This phase of work is 
more thoroughly described in the chapters \ref{usability} and 
\ref{feasibility}. In this part of the work we aim to answer 
\hyperref[rq3]{\textbf{RQ3}}.
